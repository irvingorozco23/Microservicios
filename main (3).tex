\documentclass[a4paper,12pt]{article}
\usepackage[utf8]{inputenc}
\usepackage[spanish]{babel}
\usepackage{graphicx}
\usepackage{geometry}
\usepackage{hyperref}
\usepackage{listings}
\usepackage{xcolor}
\usepackage{float}

% Configuración de márgenes
\geometry{top=2.5cm, bottom=2.5cm, left=2.5cm, right=2.5cm}

% Configuración para mostrar código
\definecolor{codegreen}{rgb}{0,0.6,0}
\definecolor{codegray}{rgb}{0.5,0.5,0.5}
\definecolor{codepurple}{rgb}{0.58,0,0.82}
\definecolor{backcolour}{rgb}{0.95,0.95,0.92}

\lstdefinestyle{mystyle}{
    backgroundcolor=\color{backcolour},   
    commentstyle=\color{codegreen},
    keywordstyle=\color{magenta},
    numberstyle=\tiny\color{codegray},
    stringstyle=\color{codepurple},
    basicstyle=\ttfamily\footnotesize,
    breakatwhitespace=false,         
    breaklines=true,                 
    captionpos=b,                    
    keepspaces=true,                 
    numbers=left,                    
    numbersep=5pt,                  
    showspaces=false,                
    showstringspaces=false,
    showtabs=false,                  
    tabsize=2
}

\lstset{style=mystyle}

% Información del documento
\title{\textbf{Documentación del Microservicio de Usuarios (FastAPI)}}
\author{Equipo de Desarrollo}
\date{\today}

\begin{document}

\maketitle
\tableofcontents
\newpage

\section{Introducción}
Este proyecto es un microservicio desarrollado en Python utilizando el framework FastAPI. Su objetivo principal es la gestión de usuarios, permitiendo operaciones básicas como crear y leer información de usuarios a través de una API RESTful.

El proyecto sigue una arquitectura en capas (similar a la arquitectura hexagonal o limpia) para separar la lógica de negocio, la infraestructura y las interfaces externas, lo que facilita el mantenimiento y la escalabilidad.

\section{Estructura del Proyecto}
El proyecto está organizado en módulos que representan diferentes capas de responsabilidad.

% IMAGEN 4: Árbol de directorios
\begin{figure}[H]
    \centering
    % Asegúrate de que el nombre coincida con tu archivo subido
    \includegraphics[width=0.6\textwidth]{estructura_proyecto.png}
    \caption{Estructura de directorios del proyecto en Visual Studio Code.}
    \label{fig:estructura}
\end{figure}

\subsection{Descripción de directorios clave}
\begin{itemize}
    \item \textbf{\texttt{app/domain/core}}: Contiene el núcleo de la lógica de negocio y las definiciones de datos (modelos). Aquí se define qué es un "Usuario" independientemente de la base de datos.
    \item \textbf{\texttt{app/application}}: (Puertos) Define las interfaces que la aplicación utiliza para comunicarse con el exterior.
    \item \textbf{\texttt{app/infraestructura}}:
    \begin{itemize}
        \item \textbf{\texttt{adapters}}: Implementaciones concretas de las interfaces (ej. conexión a base de datos).
        \item \textbf{\texttt{api}}: Contiene los controladores (\texttt{controller.py}) que manejan las peticiones HTTP.
    \end{itemize}
    \item \textbf{Archivos Raíz}:
    \begin{itemize}
        \item \texttt{main.py}: Punto de entrada principal.
        \item \texttt{Dockerfile}: Configuración para contenedorizar la aplicación.
        \item \texttt{requirements.txt}: Lista de dependencias.
    \end{itemize}
\end{itemize}

\section{Componentes Clave del Código}

\subsection{Modelo de Datos (\texttt{models.py})}
Este archivo define la estructura de los datos del usuario utilizando \textbf{Pydantic} para la validación de tipos.

% IMAGEN 6: Código de models.py
\begin{figure}[H]
    \centering
    \includegraphics[width=0.9\textwidth]{codigo_modelos.png}
    \caption{Definición de la clase Usuario en \texttt{models.py}.}
\end{figure}

La clase \texttt{Usuario} contiene los campos \texttt{idusuario} (opcional), \texttt{nombre} y \texttt{email}.

\subsection{Controlador de la API (\texttt{controller.py})}
Este archivo define las rutas (endpoints) relacionadas con los usuarios y maneja la lógica de las peticiones HTTP.

% IMAGEN 7: Código de controller.py
\begin{figure}[H]
    \centering
    \includegraphics[width=0.9\textwidth]{codigo_controller.png}
    \caption{Endpoints definidos en \texttt{controller.py}.}
\end{figure}

Se utiliza \texttt{APIRouter} para agrupar las rutas:
\begin{itemize}
    \item \textbf{POST /}: Crea un usuario y lo guarda en una lista temporal.
    \item \textbf{GET /\{idusuario\}}: Devuelve los datos de un usuario específico.
\end{itemize}

\subsection{Punto de Entrada (\texttt{main.py})}
Inicializa la aplicación FastAPI y conecta los routers.

% IMAGEN 8: Código de main.py
\begin{figure}[H]
    \centering
    \includegraphics[width=0.9\textwidth]{codigo_main.png}
    \caption{Configuración principal en \texttt{main.py}.}
\end{figure}

\section{Acceso a la API y Documentación}
FastAPI genera documentación automática en \texttt{/docs}.

\subsection{Vista General (Swagger UI)}
La interfaz muestra todos los endpoints disponibles y sus esquemas.

% IMAGEN 3: Swagger general
\begin{figure}[H]
    \centering
    \includegraphics[width=1.0\textwidth]{swagger_general.png}
    \caption{Interfaz de Swagger UI con los endpoints disponibles.}
\end{figure}

\subsection{Prueba de Endpoint (POST)}
Ejecución exitosa de creación de usuario.

% IMAGEN 2: Resultado del POST
\begin{figure}[H]
    \centering
    \includegraphics[width=1.0\textwidth]{swagger_post.png}
    \caption{Respuesta exitosa (201 Created) al crear un usuario.}
\end{figure}

% --- NUEVA SECCIÓN AÑADIDA ---
\section{Repositorio del Proyecto}

El código fuente completo de este microservicio, incluyendo la configuración de Docker y los archivos de entorno, se encuentra disponible en el siguiente repositorio de GitHub:

\begin{center}
    % SUSTITUYE EL ENLACE DE ABAJO POR EL TUYO
    \url{https://github.com/irvingorozco23/Microservicios.git}
\end{center}

\end{document}